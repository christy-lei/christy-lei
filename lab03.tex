\documentclass[]{article}
\usepackage{lmodern}
\usepackage{amssymb,amsmath}
\usepackage{ifxetex,ifluatex}
\usepackage{fixltx2e} % provides \textsubscript
\ifnum 0\ifxetex 1\fi\ifluatex 1\fi=0 % if pdftex
  \usepackage[T1]{fontenc}
  \usepackage[utf8]{inputenc}
\else % if luatex or xelatex
  \ifxetex
    \usepackage{mathspec}
  \else
    \usepackage{fontspec}
  \fi
  \defaultfontfeatures{Ligatures=TeX,Scale=MatchLowercase}
\fi
% use upquote if available, for straight quotes in verbatim environments
\IfFileExists{upquote.sty}{\usepackage{upquote}}{}
% use microtype if available
\IfFileExists{microtype.sty}{%
\usepackage[]{microtype}
\UseMicrotypeSet[protrusion]{basicmath} % disable protrusion for tt fonts
}{}
\PassOptionsToPackage{hyphens}{url} % url is loaded by hyperref
\usepackage[unicode=true]{hyperref}
\PassOptionsToPackage{usenames,dvipsnames}{color} % color is loaded by hyperref
\hypersetup{
            pdftitle={Lab 3},
            pdfauthor={Christy Lei},
            colorlinks=true,
            linkcolor=Maroon,
            citecolor=Blue,
            urlcolor=blue,
            breaklinks=true}
\urlstyle{same}  % don't use monospace font for urls
\usepackage[margin=1in]{geometry}
\usepackage{color}
\usepackage{fancyvrb}
\newcommand{\VerbBar}{|}
\newcommand{\VERB}{\Verb[commandchars=\\\{\}]}
\DefineVerbatimEnvironment{Highlighting}{Verbatim}{commandchars=\\\{\}}
% Add ',fontsize=\small' for more characters per line
\usepackage{framed}
\definecolor{shadecolor}{RGB}{248,248,248}
\newenvironment{Shaded}{\begin{snugshade}}{\end{snugshade}}
\newcommand{\KeywordTok}[1]{\textcolor[rgb]{0.13,0.29,0.53}{\textbf{#1}}}
\newcommand{\DataTypeTok}[1]{\textcolor[rgb]{0.13,0.29,0.53}{#1}}
\newcommand{\DecValTok}[1]{\textcolor[rgb]{0.00,0.00,0.81}{#1}}
\newcommand{\BaseNTok}[1]{\textcolor[rgb]{0.00,0.00,0.81}{#1}}
\newcommand{\FloatTok}[1]{\textcolor[rgb]{0.00,0.00,0.81}{#1}}
\newcommand{\ConstantTok}[1]{\textcolor[rgb]{0.00,0.00,0.00}{#1}}
\newcommand{\CharTok}[1]{\textcolor[rgb]{0.31,0.60,0.02}{#1}}
\newcommand{\SpecialCharTok}[1]{\textcolor[rgb]{0.00,0.00,0.00}{#1}}
\newcommand{\StringTok}[1]{\textcolor[rgb]{0.31,0.60,0.02}{#1}}
\newcommand{\VerbatimStringTok}[1]{\textcolor[rgb]{0.31,0.60,0.02}{#1}}
\newcommand{\SpecialStringTok}[1]{\textcolor[rgb]{0.31,0.60,0.02}{#1}}
\newcommand{\ImportTok}[1]{#1}
\newcommand{\CommentTok}[1]{\textcolor[rgb]{0.56,0.35,0.01}{\textit{#1}}}
\newcommand{\DocumentationTok}[1]{\textcolor[rgb]{0.56,0.35,0.01}{\textbf{\textit{#1}}}}
\newcommand{\AnnotationTok}[1]{\textcolor[rgb]{0.56,0.35,0.01}{\textbf{\textit{#1}}}}
\newcommand{\CommentVarTok}[1]{\textcolor[rgb]{0.56,0.35,0.01}{\textbf{\textit{#1}}}}
\newcommand{\OtherTok}[1]{\textcolor[rgb]{0.56,0.35,0.01}{#1}}
\newcommand{\FunctionTok}[1]{\textcolor[rgb]{0.00,0.00,0.00}{#1}}
\newcommand{\VariableTok}[1]{\textcolor[rgb]{0.00,0.00,0.00}{#1}}
\newcommand{\ControlFlowTok}[1]{\textcolor[rgb]{0.13,0.29,0.53}{\textbf{#1}}}
\newcommand{\OperatorTok}[1]{\textcolor[rgb]{0.81,0.36,0.00}{\textbf{#1}}}
\newcommand{\BuiltInTok}[1]{#1}
\newcommand{\ExtensionTok}[1]{#1}
\newcommand{\PreprocessorTok}[1]{\textcolor[rgb]{0.56,0.35,0.01}{\textit{#1}}}
\newcommand{\AttributeTok}[1]{\textcolor[rgb]{0.77,0.63,0.00}{#1}}
\newcommand{\RegionMarkerTok}[1]{#1}
\newcommand{\InformationTok}[1]{\textcolor[rgb]{0.56,0.35,0.01}{\textbf{\textit{#1}}}}
\newcommand{\WarningTok}[1]{\textcolor[rgb]{0.56,0.35,0.01}{\textbf{\textit{#1}}}}
\newcommand{\AlertTok}[1]{\textcolor[rgb]{0.94,0.16,0.16}{#1}}
\newcommand{\ErrorTok}[1]{\textcolor[rgb]{0.64,0.00,0.00}{\textbf{#1}}}
\newcommand{\NormalTok}[1]{#1}
\usepackage{graphicx,grffile}
\makeatletter
\def\maxwidth{\ifdim\Gin@nat@width>\linewidth\linewidth\else\Gin@nat@width\fi}
\def\maxheight{\ifdim\Gin@nat@height>\textheight\textheight\else\Gin@nat@height\fi}
\makeatother
% Scale images if necessary, so that they will not overflow the page
% margins by default, and it is still possible to overwrite the defaults
% using explicit options in \includegraphics[width, height, ...]{}
\setkeys{Gin}{width=\maxwidth,height=\maxheight,keepaspectratio}
\IfFileExists{parskip.sty}{%
\usepackage{parskip}
}{% else
\setlength{\parindent}{0pt}
\setlength{\parskip}{6pt plus 2pt minus 1pt}
}
\setlength{\emergencystretch}{3em}  % prevent overfull lines
\providecommand{\tightlist}{%
  \setlength{\itemsep}{0pt}\setlength{\parskip}{0pt}}
\setcounter{secnumdepth}{0}
% Redefines (sub)paragraphs to behave more like sections
\ifx\paragraph\undefined\else
\let\oldparagraph\paragraph
\renewcommand{\paragraph}[1]{\oldparagraph{#1}\mbox{}}
\fi
\ifx\subparagraph\undefined\else
\let\oldsubparagraph\subparagraph
\renewcommand{\subparagraph}[1]{\oldsubparagraph{#1}\mbox{}}
\fi

% set default figure placement to htbp
\makeatletter
\def\fps@figure{htbp}
\makeatother


\title{Lab 3}
\author{Christy Lei}
\date{Math 241, Week 4}

\begin{document}
\maketitle

\begin{Shaded}
\begin{Highlighting}[]
\CommentTok{# Put all necessary libraries here}
\KeywordTok{library}\NormalTok{(tidyverse)}
\KeywordTok{library}\NormalTok{(ggplot2)}
\KeywordTok{library}\NormalTok{(ggthemes)}
\end{Highlighting}
\end{Shaded}

\subsection{Due: Thursday, February 27th at
8:30am}\label{due-thursday-february-27th-at-830am}

\subsection{Goals of this lab}\label{goals-of-this-lab}

\begin{enumerate}
\def\labelenumi{\arabic{enumi}.}
\tightlist
\item
  Practice using GitHub.
\item
  Practice wrangling data.
\end{enumerate}

\subsection{Data Notes:}\label{data-notes}

\begin{itemize}
\tightlist
\item
  For Problem 2, we will continue to dig into the SE Portland crash data
  but will use two datasets:

  \begin{itemize}
  \tightlist
  \item
    \texttt{CRASH}: crash level data
  \item
    \texttt{PARTIC}: participant level data
  \end{itemize}
\end{itemize}

\begin{Shaded}
\begin{Highlighting}[]
\CommentTok{# Crash level dataset}
\NormalTok{crash <-}\StringTok{ }\KeywordTok{read_csv}\NormalTok{(}\StringTok{"/home/courses/math241s20/Data/pdx_crash_2018_CRASH.csv"}\NormalTok{)}

\CommentTok{# Participant level dataset}
\NormalTok{partic <-}\StringTok{ }\KeywordTok{read_csv}\NormalTok{(}\StringTok{"/home/courses/math241s20/Data/pdx_crash_2018_PARTIC.csv"}\NormalTok{)}
\end{Highlighting}
\end{Shaded}

\begin{itemize}
\tightlist
\item
  For Problem 3, we will look at chronic illness data from the
  \href{https://www.cdc.gov/cdi/index.html}{CDC} along with the regional
  mapping for each state.
\end{itemize}

\begin{Shaded}
\begin{Highlighting}[]
\CommentTok{# CDC data}
\NormalTok{CDC <-}\StringTok{ }\KeywordTok{read_csv}\NormalTok{(}\StringTok{"/home/courses/math241s20/Data/CDC2.csv"}\NormalTok{)}

\CommentTok{# Regional data}
\NormalTok{USregions <-}\StringTok{ }\KeywordTok{read_csv}\NormalTok{(}\StringTok{"/home/courses/math241s20/Data/USregions.csv"}\NormalTok{)}
\end{Highlighting}
\end{Shaded}

\begin{itemize}
\tightlist
\item
  For Problem 4, we will use polling data from
  \href{https://projects.fivethirtyeight.com/congress-generic-ballot-polls/}{FiveThirtyEight.com}.
\end{itemize}

\begin{Shaded}
\begin{Highlighting}[]
\CommentTok{# Note I only want us to focus on a subset of the variables}
\NormalTok{polls <-}\StringTok{ }\KeywordTok{read_csv}\NormalTok{(}\StringTok{"/home/courses/math241s20/Data/generic_topline.csv"}\NormalTok{) }\OperatorTok
\StringTok{  }\KeywordTok{select}\NormalTok{(subgroup, modeldate, dem_estimate, rep_estimate)}
\end{Highlighting}
\end{Shaded}

\subsection{Problems}\label{problems}

\subsubsection{Problem 1: Git Control}\label{problem-1-git-control}

In this problem, we will practice interacting with GitHub on the site
directly and from the RStudio Server. Do this practice on \textbf{your
repo}, not your group's Project 1 repo, so that the graders can check
your progress with Git.

\begin{enumerate}
\def\labelenumi{\alph{enumi}.}
\item
  Let's practice creating and closing \textbf{Issues}. In a nutshell,
  \textbf{Issues} let us keep track of our work. Within your repo on
  GitHub.com, create an Issue entitled ``Complete Lab 3''. Once Lab 3 is
  done, close the \textbf{Issue}. (If you want to learn more about the
  functionalities of Issues, check out this
  \href{https://guides.github.com/features/issues/}{page}.)
\item
  Edit the ReadMe of your repo to include your name and a quick summary
  of the purpose of the repo. You can edit from within GitHub directly
  or on the server. If you edit on the server, make sure to push your
  changes to GitHub.
\item
  Upload both your Lab 3 .Rmd and .pdf to your repo on GitHub.
\end{enumerate}

\subsubsection{\texorpdfstring{Problem 2: \texttt{dplyr}
madness}{Problem 2: dplyr madness}}\label{problem-2-dplyr-madness}

Each part of this problem will require you to wrangle the data and then
do one or both of the following:

\begin{itemize}
\tightlist
\item
  Display the wrangled data frame. To ensure it displays the whole data
  frame, you can pipe \texttt{as.data.frame()} at the end of the
  wrangling.
\item
  Answer a question(s).
\end{itemize}

\textbf{Some parts will require you to do a data join but won't tell you
that.}

\begin{enumerate}
\def\labelenumi{\alph{enumi}.}
\tightlist
\item
  Produce a table that provides the frequency of the different collision
  types, ordered from most to least common. What type is most common?
  What type is least common?
\end{enumerate}

\begin{Shaded}
\begin{Highlighting}[]
\NormalTok{crash }\OperatorTok\StringTok{ }
\StringTok{  }\KeywordTok{count}\NormalTok{(COLLIS_TYP_CD) }\OperatorTok\StringTok{ }
\StringTok{  }\KeywordTok{arrange}\NormalTok{(}\KeywordTok{desc}\NormalTok{(n)) }\OperatorTok\StringTok{ }
\StringTok{  }\KeywordTok{as.data.frame}\NormalTok{()}
\end{Highlighting}
\end{Shaded}

\begin{verbatim}
##    COLLIS_TYP_CD   n
## 1              3 671
## 2              6 365
## 3              1 241
## 4              5  89
## 5              0  86
## 6              9  51
## 7              4  17
## 8              2  16
## 9              -  12
## 10             7  10
## 11             8   6
## 12             &   3
\end{verbatim}

Rear-End (type 3) is the most common collision type and Miscellaneous
(type \&) is the least common one.

\begin{enumerate}
\def\labelenumi{\alph{enumi}.}
\setcounter{enumi}{1}
\tightlist
\item
  For the three most common collision types, create a table that
  contains:

  \begin{itemize}
  \tightlist
  \item
    The frequencies of each collision type and weather condition
    combination.
  \item
    The proportion of each collision type by weather condition.
  \end{itemize}
\end{enumerate}

Arrange the table by weather and within type, most to least common
collision type.

\begin{Shaded}
\begin{Highlighting}[]
\NormalTok{crash }\OperatorTok\StringTok{ }
\StringTok{  }\KeywordTok{count}\NormalTok{(COLLIS_TYP_CD, WTHR_COND_CD) }\OperatorTok
\StringTok{  }\KeywordTok{group_by}\NormalTok{(WTHR_COND_CD) }\OperatorTok\StringTok{ }
\StringTok{  }\KeywordTok{mutate}\NormalTok{(}\DataTypeTok{prop =} \KeywordTok{prop.table}\NormalTok{(n)) }\OperatorTok
\StringTok{  }\KeywordTok{arrange}\NormalTok{(}\KeywordTok{desc}\NormalTok{(n)) }\OperatorTok\StringTok{ }
\StringTok{  }\KeywordTok{as.data.frame}\NormalTok{()}
\end{Highlighting}
\end{Shaded}

\begin{verbatim}
##    COLLIS_TYP_CD WTHR_COND_CD   n        prop
## 1              3            1 549 0.447797716
## 2              6            1 290 0.236541599
## 3              1            1 188 0.153344209
## 4              5            1  73 0.059543230
## 5              3            3  71 0.381720430
## 6              0            1  50 0.040783034
## 7              6            3  44 0.236559140
## 8              3            2  29 0.308510638
## 9              1            3  28 0.150537634
## 10             9            1  28 0.022838499
## 11             3            0  20 0.434782609
## 12             6            2  20 0.212765957
## 13             0            3  17 0.091397849
## 14             0            2  16 0.170212766
## 15             4            1  15 0.012234910
## 16             1            2  13 0.138297872
## 17             2            1  12 0.009787928
## 18             9            3  11 0.059139785
## 19             -            1   8 0.006525285
## 20             6            0   8 0.173913043
## 21             7            1   8 0.006525285
## 22             9            2   8 0.085106383
## 23             1            0   6 0.130434783
## 24             5            3   6 0.032258065
## 25             5            2   5 0.053191489
## 26             5            0   4 0.086956522
## 27             -            3   3 0.016129032
## 28             &            1   3 0.002446982
## 29             1            6   3 0.333333333
## 30             9            0   3 0.065217391
## 31             0            0   2 0.043478261
## 32             1            5   2 0.666666667
## 33             4            3   2 0.010752688
## 34             6            6   2 0.222222222
## 35             8            1   2 0.001631321
## 36             8            2   2 0.021276596
## 37             8            3   2 0.010752688
## 38             -            0   1 0.021739130
## 39             0            8   1 0.500000000
## 40             1            4   1 1.000000000
## 41             2            0   1 0.021739130
## 42             2            2   1 0.010638298
## 43             2            3   1 0.005376344
## 44             2            6   1 0.111111111
## 45             3            6   1 0.111111111
## 46             3            8   1 0.500000000
## 47             5            6   1 0.111111111
## 48             6            5   1 0.333333333
## 49             7            0   1 0.021739130
## 50             7            3   1 0.005376344
## 51             9            6   1 0.111111111
\end{verbatim}

\begin{enumerate}
\def\labelenumi{\alph{enumi}.}
\setcounter{enumi}{2}
\tightlist
\item
  Create a column for whether or not a crash happened on a weekday or on
  the weekend and then create a data frame that explores if the
  distribution of collision types varies by whether or not the crash
  happened during the week or the weekend.
\end{enumerate}

\begin{Shaded}
\begin{Highlighting}[]
\NormalTok{weekday =}\StringTok{ }\KeywordTok{c}\NormalTok{(}\DecValTok{1}\NormalTok{,}\DecValTok{2}\NormalTok{,}\DecValTok{3}\NormalTok{,}\DecValTok{4}\NormalTok{,}\DecValTok{5}\NormalTok{)}
\NormalTok{weekend =}\StringTok{ }\KeywordTok{c}\NormalTok{(}\DecValTok{6}\NormalTok{,}\DecValTok{7}\NormalTok{)}
\NormalTok{crash}\OperatorTok{$}\NormalTok{weekday_true =}\StringTok{ }
\StringTok{  }\KeywordTok{ifelse}\NormalTok{(crash}\OperatorTok{$}\NormalTok{CRASH_WK_DAY_CD }\OperatorTok\StringTok{ }\NormalTok{weekday, }\StringTok{"Yes"}\NormalTok{,}
         \KeywordTok{ifelse}\NormalTok{(crash}\OperatorTok{$}\NormalTok{CRASH_WK_DAY_CD }\OperatorTok\StringTok{ }\NormalTok{weekend, }\StringTok{"No"}\NormalTok{,}
                \OtherTok{NA}\NormalTok{))}

\NormalTok{crash_weekday <-}\StringTok{ }\NormalTok{crash }\OperatorTok\StringTok{ }
\StringTok{  }\KeywordTok{count}\NormalTok{(COLLIS_TYP_CD, weekday_true) }\OperatorTok
\StringTok{  }\KeywordTok{group_by}\NormalTok{(weekday_true) }\OperatorTok
\StringTok{  }\KeywordTok{mutate}\NormalTok{(}\DataTypeTok{prop =} \KeywordTok{prop.table}\NormalTok{(n)) }\OperatorTok
\StringTok{  }\KeywordTok{arrange}\NormalTok{(}\KeywordTok{desc}\NormalTok{(n))}

\NormalTok{crash_weekday}
\end{Highlighting}
\end{Shaded}

\begin{verbatim}
## # A tibble: 23 x 4
## # Groups:   weekday_true [2]
##    COLLIS_TYP_CD weekday_true     n   prop
##    <chr>         <chr>        <int>  <dbl>
##  1 3             Yes            461 0.419 
##  2 6             Yes            256 0.233 
##  3 3             No             210 0.450 
##  4 1             Yes            172 0.156 
##  5 6             No             109 0.233 
##  6 1             No              69 0.148 
##  7 0             Yes             61 0.0555
##  8 5             Yes             60 0.0545
##  9 9             Yes             40 0.0364
## 10 5             No              29 0.0621
## # ... with 13 more rows
\end{verbatim}

It seems that the most common collision type (Type 3) happened mostly on
weekdays.

\begin{enumerate}
\def\labelenumi{\alph{enumi}.}
\setcounter{enumi}{3}
\tightlist
\item
  First determine what proportion of crashes involve pedestrians. Then,
  for each driver license status, determine what proportion of crashes
  involve pedestrians. What driver license status has the highest rate
  of crashes that involve pedestrians?
\end{enumerate}

\begin{Shaded}
\begin{Highlighting}[]
\CommentTok{#proportion of crashes that involve pedestrians}
\NormalTok{crash }\OperatorTok\StringTok{ }
\StringTok{  }\KeywordTok{count}\NormalTok{(}\DataTypeTok{pedestrians_involved =}\NormalTok{ (CRASH_TYP_CD }\OperatorTok{==}\StringTok{ }\DecValTok{3}\NormalTok{)) }\OperatorTok
\StringTok{  }\KeywordTok{mutate}\NormalTok{(}\DataTypeTok{prop =} \KeywordTok{prop.table}\NormalTok{(n))}
\end{Highlighting}
\end{Shaded}

\begin{verbatim}
## # A tibble: 2 x 3
##   pedestrians_involved     n   prop
##   <lgl>                <int>  <dbl>
## 1 FALSE                 1481 0.945 
## 2 TRUE                    86 0.0549
\end{verbatim}

\begin{Shaded}
\begin{Highlighting}[]
\CommentTok{#proportion of crashes that involve pedestrians for each driver license status}
\NormalTok{crash_and_partic <-}\StringTok{ }\KeywordTok{left_join}\NormalTok{ (crash, partic, }\DataTypeTok{by =} \KeywordTok{c}\NormalTok{(}\StringTok{"CRASH_ID"}\NormalTok{ =}\StringTok{ "CRASH_ID"}\NormalTok{)) }
\NormalTok{crash_and_partic }\OperatorTok
\StringTok{  }\KeywordTok{group_by}\NormalTok{(DRVR_LIC_STAT_CD) }\OperatorTok
\StringTok{  }\KeywordTok{count}\NormalTok{(}\DataTypeTok{pedestrians_involved =}\NormalTok{ (CRASH_TYP_CD }\OperatorTok{==}\StringTok{ }\DecValTok{3}\NormalTok{)) }\OperatorTok
\StringTok{  }\KeywordTok{mutate}\NormalTok{(}\DataTypeTok{prop =} \KeywordTok{prop.table}\NormalTok{(n)) }\OperatorTok
\StringTok{  }\KeywordTok{arrange}\NormalTok{(}\KeywordTok{desc}\NormalTok{(n))}
\end{Highlighting}
\end{Shaded}

\begin{verbatim}
## # A tibble: 12 x 4
## # Groups:   DRVR_LIC_STAT_CD [7]
##    DRVR_LIC_STAT_CD pedestrians_involved     n    prop
##               <dbl> <lgl>                <int>   <dbl>
##  1                1 FALSE                 2410 0.971  
##  2               NA FALSE                  795 0.991  
##  3                2 FALSE                  305 0.981  
##  4                9 FALSE                  258 0.992  
##  5                1 TRUE                    72 0.0290 
##  6                3 FALSE                   51 0.879  
##  7                0 FALSE                   24 1      
##  8                3 TRUE                     7 0.121  
##  9               NA TRUE                     7 0.00873
## 10                2 TRUE                     6 0.0193 
## 11                8 FALSE                    3 1      
## 12                9 TRUE                     2 0.00769
\end{verbatim}

5.49\% of all the crashs involve pedestrians (code 3).\n Driver license
type 1 (Valid Oregon license or permit) has the highest rate of crashs
(2.9\%) that involve pedestrians.

\begin{enumerate}
\def\labelenumi{\alph{enumi}.}
\setcounter{enumi}{4}
\tightlist
\item
  Create a data frame that contains the age of drivers and collision
  type. (Don't print it.) Complete the following:

  \begin{itemize}
  \tightlist
  \item
    Find the average and median age of drivers.
  \item
    Find the average and median age of drivers by collision type.
  \item
    Create a graph of driver ages.
  \item
    Create a graph of driver ages by collision type.
  \end{itemize}
\end{enumerate}

Draw some conclusions.

\begin{Shaded}
\begin{Highlighting}[]
\NormalTok{collision_type_age <-}\StringTok{ }\NormalTok{crash_and_partic }\OperatorTok\StringTok{ }
\StringTok{  }\KeywordTok{select}\NormalTok{(COLLIS_TYP_CD, AGE_VAL) }

\CommentTok{# Find the average and median age of drivers}
\NormalTok{collision_type_age }\OperatorTok
\StringTok{  }\KeywordTok{summarize}\NormalTok{(}\DataTypeTok{mean_age =} \KeywordTok{mean}\NormalTok{(}\KeywordTok{as.numeric}\NormalTok{(AGE_VAL)), }
            \DataTypeTok{median_age =} \KeywordTok{median}\NormalTok{(}\KeywordTok{as.numeric}\NormalTok{(AGE_VAL)))}
\end{Highlighting}
\end{Shaded}

\begin{verbatim}
## # A tibble: 1 x 2
##   mean_age median_age
##      <dbl>      <dbl>
## 1     34.9         34
\end{verbatim}

\begin{Shaded}
\begin{Highlighting}[]
\CommentTok{# Find the average and median age of drivers by collision type}
\NormalTok{collision_type_age }\OperatorTok
\StringTok{  }\KeywordTok{group_by}\NormalTok{(COLLIS_TYP_CD) }\OperatorTok
\StringTok{  }\KeywordTok{summarize}\NormalTok{(}\DataTypeTok{mean_age =} \KeywordTok{mean}\NormalTok{(}\KeywordTok{as.numeric}\NormalTok{(AGE_VAL)), }
            \DataTypeTok{median_age =} \KeywordTok{median}\NormalTok{(}\KeywordTok{as.numeric}\NormalTok{(AGE_VAL)))}
\end{Highlighting}
\end{Shaded}

\begin{verbatim}
## # A tibble: 12 x 3
##    COLLIS_TYP_CD mean_age median_age
##    <chr>            <dbl>      <dbl>
##  1 -                 36.0         37
##  2 &                 49.1         40
##  3 0                 44.7         44
##  4 1                 36.6         36
##  5 2                 35.8         29
##  6 3                 34.0         33
##  7 4                 35.8         33
##  8 5                 33.3         33
##  9 6                 34.7         33
## 10 7                 35.6         39
## 11 8                 37.4         36
## 12 9                 34.8         31
\end{verbatim}

\begin{Shaded}
\begin{Highlighting}[]
\CommentTok{# Create a graph of driver ages}
\NormalTok{collision_type_age }\OperatorTok\StringTok{ }
\StringTok{  }\KeywordTok{ggplot}\NormalTok{(}\KeywordTok{aes}\NormalTok{(}\DataTypeTok{x =}\NormalTok{ AGE_VAL)) }\OperatorTok{+}
\StringTok{  }\KeywordTok{geom_bar}\NormalTok{()}
\end{Highlighting}
\end{Shaded}

\includegraphics{lab03_files/figure-latex/unnamed-chunk-9-1.pdf}

\begin{Shaded}
\begin{Highlighting}[]
\CommentTok{# Create a graph of driver ages by collision type}
\NormalTok{collision_type_age }\OperatorTok\StringTok{ }
\StringTok{  }\KeywordTok{ggplot}\NormalTok{(}\KeywordTok{aes}\NormalTok{(}\DataTypeTok{x =}\NormalTok{ AGE_VAL, }\DataTypeTok{fill =}\NormalTok{ COLLIS_TYP_CD)) }\OperatorTok{+}
\StringTok{  }\KeywordTok{geom_dotplot}\NormalTok{(}\DataTypeTok{binwidth =} \FloatTok{0.8}\NormalTok{) }\OperatorTok{+}
\StringTok{  }\KeywordTok{scale_fill_brewer}\NormalTok{(}\DataTypeTok{palette =} \StringTok{"Accent"}\NormalTok{) }\OperatorTok{+}
\StringTok{  }\KeywordTok{theme_fivethirtyeight}\NormalTok{() }\OperatorTok{+}
\StringTok{  }\KeywordTok{theme}\NormalTok{(}\DataTypeTok{panel.grid =} \KeywordTok{element_blank}\NormalTok{(),}
        \DataTypeTok{plot.title =} \KeywordTok{element_text}\NormalTok{(}\DataTypeTok{size =} \DecValTok{16}\NormalTok{)) }\OperatorTok{+}
\StringTok{   }\KeywordTok{ylim}\NormalTok{(}\DecValTok{0}\NormalTok{,}\DecValTok{900}\NormalTok{) }\OperatorTok{+}\StringTok{ }
\StringTok{  }\KeywordTok{labs}\NormalTok{(}\DataTypeTok{title =} \StringTok{"Driver ages by collision type"}\NormalTok{)}
\end{Highlighting}
\end{Shaded}

\includegraphics{lab03_files/figure-latex/unnamed-chunk-9-2.pdf}

\subsubsection{Problem 3: Chronically Messy
Data}\label{problem-3-chronically-messy-data}

\begin{enumerate}
\def\labelenumi{\alph{enumi}.}
\tightlist
\item
  Turning to the CDC data, let's get a handle of what is represented
  there. For 2016 (use \texttt{YearStart}), how many distinct topics
  were tracked?
\end{enumerate}

\begin{Shaded}
\begin{Highlighting}[]
\NormalTok{CDC }\OperatorTok\StringTok{ }
\StringTok{  }\KeywordTok{filter}\NormalTok{(YearStart }\OperatorTok{==}\StringTok{ }\DecValTok{2016}\NormalTok{) }\OperatorTok
\StringTok{  }\KeywordTok{count}\NormalTok{(Topic) }\OperatorTok
\StringTok{  }\KeywordTok{nrow}\NormalTok{()}
\end{Highlighting}
\end{Shaded}

\begin{verbatim}
## [1] 16
\end{verbatim}

16 distinct topics were tracked.

\begin{enumerate}
\def\labelenumi{\alph{enumi}.}
\setcounter{enumi}{1}
\item
  Let's study influenza vaccination patterns! Create a dataset that
  contains the age adjusted prevalence of the ``Influenza vaccination
  among noninstitutionalized adults aged \textgreater{}= 18 years'' for
  Oregon and the US from 2010 to 2016.
\item
  Create a graph comparing the immunization rates of Pennsylvania and
  the US. Comment on the observed trends in your graph
\item
  Let's see how immunization rates vary by region of the country. Join
  the regional dataset to our CDC dataset so that we have a column
  signifying the region of the country.
\item
  Why are there NAs in the region column of the new dataset?
\item
  Create a dataset that contains the age adjusted influenza immunization
  rates in 2016 for each state in the country and sort it by highest
  immunization to lowest. Which state has the highest immunization?
\item
  Construct a graphic of the 2016 influenza immunization rates by region
  of the country. Don't include locations without a region. Comment on
  your graphic.
\end{enumerate}

\subsubsection{Problem 4: Tidying Data Like a
Boss}\label{problem-4-tidying-data-like-a-boss}

I was amazed by the fact that many of the FiveThirtyEight datasets are
actually not in a perfectly \emph{tidy} format. Let's tidy up this
dataset related to
\href{https://projects.fivethirtyeight.com/congress-generic-ballot-polls/}{polling}.

\begin{enumerate}
\def\labelenumi{\alph{enumi}.}
\tightlist
\item
  Why is this data not currently in a tidy format?
\end{enumerate}

\begin{Shaded}
\begin{Highlighting}[]
\NormalTok{polls}
\end{Highlighting}
\end{Shaded}

\begin{verbatim}
## # A tibble: 1,529 x 4
##    subgroup  modeldate dem_estimate rep_estimate
##    <chr>     <chr>            <dbl>        <dbl>
##  1 All polls 9/18/2018         48.8         39.8
##  2 All polls 9/17/2018         49.0         39.9
##  3 All polls 9/16/2018         49.0         39.9
##  4 All polls 9/15/2018         49.0         39.9
##  5 All polls 9/14/2018         48.9         39.8
##  6 All polls 9/13/2018         48.8         39.7
##  7 All polls 9/12/2018         48.8         39.6
##  8 All polls 9/11/2018         48.5         39.9
##  9 All polls 9/10/2018         48.4         39.9
## 10 All polls 9/9/2018          48.4         39.9
## # ... with 1,519 more rows
\end{verbatim}

\begin{enumerate}
\def\labelenumi{\alph{enumi}.}
\setcounter{enumi}{1}
\item
  Create a tidy dataset of the \texttt{All\ polls} subgroup.
\item
  Now let's create a new untidy version of \texttt{polls}. Focusing just
  on the estimates for democrats, create a data frame where each row
  represents a subgroup (given in column 1) and the rest of the columns
  are the estimates for democrats by date.
\item
  Why might someone want to transform the data like we did in part c?
\end{enumerate}

\subsubsection{Problem 5: YOUR TURN!}\label{problem-5-your-turn}

Now it is your turn. Pick one (or multiple) of the datasets used on this
lab. Ask a question of the data. Do some data wrangling to produce
statistics (use at least two wrangling verbs) and a graphic to answer
the question. Then comment on any conclusions you can draw about your
question.

\end{document}
